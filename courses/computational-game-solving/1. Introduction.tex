% Options for packages loaded elsewhere
\PassOptionsToPackage{unicode}{hyperref}
\PassOptionsToPackage{hyphens}{url}
%
\documentclass[
]{article}
\usepackage{amsmath,amssymb}
\usepackage{iftex}
\ifPDFTeX
  \usepackage[T1]{fontenc}
  \usepackage[utf8]{inputenc}
  \usepackage{textcomp} % provide euro and other symbols
\else % if luatex or xetex
  \usepackage{unicode-math} % this also loads fontspec
  \defaultfontfeatures{Scale=MatchLowercase}
  \defaultfontfeatures[\rmfamily]{Ligatures=TeX,Scale=1}
\fi
\usepackage{lmodern}
\ifPDFTeX\else
  % xetex/luatex font selection
\fi
% Use upquote if available, for straight quotes in verbatim environments
\IfFileExists{upquote.sty}{\usepackage{upquote}}{}
\IfFileExists{microtype.sty}{% use microtype if available
  \usepackage[]{microtype}
  \UseMicrotypeSet[protrusion]{basicmath} % disable protrusion for tt fonts
}{}
\makeatletter
\@ifundefined{KOMAClassName}{% if non-KOMA class
  \IfFileExists{parskip.sty}{%
    \usepackage{parskip}
  }{% else
    \setlength{\parindent}{0pt}
    \setlength{\parskip}{6pt plus 2pt minus 1pt}}
}{% if KOMA class
  \KOMAoptions{parskip=half}}
\makeatother
\usepackage{xcolor}
\usepackage{graphicx}
\makeatletter
\def\maxwidth{\ifdim\Gin@nat@width>\linewidth\linewidth\else\Gin@nat@width\fi}
\def\maxheight{\ifdim\Gin@nat@height>\textheight\textheight\else\Gin@nat@height\fi}
\makeatother
% Scale images if necessary, so that they will not overflow the page
% margins by default, and it is still possible to overwrite the defaults
% using explicit options in \includegraphics[width, height, ...]{}
\setkeys{Gin}{width=\maxwidth,height=\maxheight,keepaspectratio}
% Set default figure placement to htbp
\makeatletter
\def\fps@figure{htbp}
\makeatother
\setlength{\emergencystretch}{3em} % prevent overfull lines
\providecommand{\tightlist}{%
  \setlength{\itemsep}{0pt}\setlength{\parskip}{0pt}}
\setcounter{secnumdepth}{-\maxdimen} % remove section numbering
\ifLuaTeX
  \usepackage{selnolig}  % disable illegal ligatures
\fi
\IfFileExists{bookmark.sty}{\usepackage{bookmark}}{\usepackage{hyperref}}
\IfFileExists{xurl.sty}{\usepackage{xurl}}{} % add URL line breaks if available
\urlstyle{same}
\hypersetup{
  pdftitle={1. Introduction},
  hidelinks,
  pdfcreator={LaTeX via pandoc}}

\title{1. Introduction}
\author{}
\date{}

\begin{document}
\maketitle

\href{https://www.cs.cmu.edu/~sandholm/cs15-888F21/lecture1.pdf}{Link}

\begin{center}\rule{0.5\linewidth}{0.5pt}\end{center}

\subparagraph{1 - Introduction}\label{introduction}

\textbf{Q:} Why study multi-step imperfect-information games?

\textbf{A:} Most real-world games are incomplete-information games with
sequential (or simultaneous) moves.

Examples:

\begin{itemize}
\tightlist
\item
  Negotiation
\item
  Multi-stage auctions
\item
  Sequential auctions of multiple items
\item
  Card games
\item
  etc.
\end{itemize}

\begin{quote}
Techniques for perfect-information games (ex. checkers, chess, and go)
don\textquotesingle t apply to incomplete-information games.
\end{quote}

\textbf{Why?}

\begin{itemize}
\tightlist
\item
  Private information
\item
  Need to understand signals and how other players will interpret
  signals
\item
  Need to understand deception
\end{itemize}

\subparagraph{2 - Terminology}\label{terminology}

\begin{itemize}
\tightlist
\item
  {\emph{Agent}} (or \emph{player})\\
  \strut \\
\item
  {\emph{Action}} (or \emph{move}) - choice that the agent can make at a
  point in the game.\\
  \strut \\
\end{itemize}

\begin{itemize}
\tightlist
\item
  {\emph{Strategy}} ({}) - mapping from history to actions.\\
  {}(to the extent the agent can distinguish a history)\\
  \strut \\
\item
  {\emph{Strategy set}} ({}) - strategies available to the agent.\\
  \strut \\
\item
  {\emph{Strategy profile}} ({}) - one strategy for each agent.\\
  \strut \\
\end{itemize}

\begin{itemize}
\tightlist
\item
  {\emph{Utility}} ({}) -\\
  Represents the motivations of the agents, maps outcomes to reals with
  the property that a higher number implies that outcome is more
  preferred.\\
  \strut \\
  \emph{Note} - Utility seems to be closely related to \emph{reward} in
  RL?\\
  \strut \\
  An agent\textquotesingle s utility is only determined after all
  agents, including nature, have chosen their strategy.\\
  \strut \\
\end{itemize}

\begin{itemize}
\tightlist
\item
  {\emph{Nature}} - pseudo agent that is used to model uncertainty.\\
  \emph{Note} - Nature seems to be closely related to \emph{environment}
  in RL.
\end{itemize}

\subparagraph{3 - Agenthood}\label{agenthood}

\begin{quote}
Agents seek to maximize their expected utility:\\
\strut \\
{}\strut \\
\strut \\
\end{quote}

\begin{itemize}
\item
  Utility functions are scale-invariant.\\
  \strut \\
  \emph{Note} - We are only interested in the relative utility between
  strategies, so scaling them all by a linear transform does not change
  our rankings of actions.\\
  \strut \\
\item
  The utility function is up to the agent and can be used to model an
  agent\textquotesingle s risk attitude.

  \emph{Example:}\\
  Consider the two lotteries:\\
  Lottery 1 - 100\% chance of winning \$0.5M\\
  Lottery 2 - 50\% chance off winning \$1M

  The lotteries have equal expected values however agents may prefer one
  to the other depending on their willingness to take risk.
\end{itemize}

\begin{itemize}
\tightlist
\item
  Often times, in game theory, only expected payoff or expected value
  (EV) is considered.
\end{itemize}

\subparagraph{4 - Game representations}\label{game-representations}

\begin{itemize}
\item
  {\emph{Extensive form}} - game tree form in which agents are nodes and
  their actions are edges.\\
  \strut \\
\item
  {\emph{Matrix form}} - Player 1\textquotesingle s strategy is along
  the rows, Player 2\textquotesingle s strategy is along the columns,
  each cell denotes the utility for each player given the action pair.
\end{itemize}

\subparagraph{5 - Dominant strategy
"equilibrium"}\label{dominant-strategy-equilibrium}

\begin{itemize}
\item
  {\emph{Best response}} ({}) - {} .\\
  The strategy that yields the highest utility in response to the
  opposing player\textquotesingle s strategy, {} .\\
  \strut \\
\item
  {\emph{Dominant strategy}} ({}) - {} is the best response for all
  opponent strategies, {}\\
  The agent\textquotesingle s best strategy does not depend on its
  opponent\textquotesingle s strategy.\\
  \strut \\
\item
  {\emph{Dominant strategy "equilibrium"}} -\\
  Strategy profile in which each agent has chosen its dominant
  strategy.\\
  Therefore, no agent has motivation to change their strategy.
\end{itemize}

\emph{Note} - dominant strategies don\textquotesingle t always exist.

\subparagraph{6 - Nash equilibrium}\label{nash-equilibrium}

\begin{itemize}
\tightlist
\item
  {\emph{Nash equilibrium}} -\\
  No player has incentive to deviate from their strategy so long as
  their opponents do not deviate either. For every agent {}, {}\\
  \strut \\
\end{itemize}

\begin{quote}
Nash equilibrium is a subset of the dominant strategy equilibrium.
Dominant strategy equilibrium holds for all opponent strategies;
whereas, Nash equilibrium only holds for a fixed opponent strategy.

\emph{i.e.} I don\textquotesingle t change if you don\textquotesingle t
change, and vice versa.
\end{quote}

\hfill\break

Nash equilibrium may not exist or be unique for all games. (ex. Battle
of the Sexes)

\begin{itemize}
\item
  {\textbf{Theorem:}} \textbf{Existence of (pure-strategy) Nash
  equilibria}\\
  Any finite game, where each action node is alone in its information
  set is dominance solvable by backward induction.

  \emph{Note} - an agent is "alone in its information set" if at every
  point in the game, the agent knows what moves have been played so far.

  \textbf{Proof:} Multi-player minimax search
\end{itemize}

\subparagraph{7 - Mixed-strategy Nash equilibrium
(rock-paper-scissors)}\label{mixed-strategy-nash-equilibrium-rock-paper-scissors}

\includegraphics[width=5.20833in,height=\textheight]{C:/Users/SBC98/Desktop/Misc/obsidian-vaults/root/rock-paper-scissors.jpg}

\begin{itemize}
\item
  {\emph{Bayesian game}} - game in which players have incomplete
  information.\\
  In the case of rock-paper-scissors, the agents don\textquotesingle t
  have knowledge of the other\textquotesingle s move.\\
  \strut \\
\item
  {\emph{Bayes-Nash equilibrium}} - Nash equilibrium of a Bayesian
  game.\\
  A strategy profile that maximizes the \emph{expected} payoff for each
  player given their beliefs and strategies played by others.\\
  \strut \\
\item
  {\emph{Mixed strategy}} - the agent\textquotesingle s strategy is a
  weighted mixture of pure strategies.\\
  Rock-paper-scissors has a \emph{symmetric} mixed strategy Nash
  equilibrium where all the agents have the same strategy.\\
  \strut \\
\item
  Existence:\\
  {\textbf{Theorem:}} Every finite player, finite strategy game has at
  least one Nash equilibrium if we admit mixed-strategy equilibria as
  well as pure. {{[}1950-Nash{]}}.\\
  \strut \\
\item
  Complexity of finding a Nash equilibrium in a normal form game:

  \begin{itemize}
  \tightlist
  \item
    2-player 0-sum: polytime with LP
  \item
    2-player games:

    \begin{itemize}
    \tightlist
    \item
      PPAD-complete {{[}2009-Chen{]} {[}2005-Abbot{]}
      {[}2006-Daskalakis{]}}
    \item
      NP-complete to find an approx. \emph{good} Nash equilibrium
      {{[}2008-Conitzer{]}}
    \end{itemize}
  \item
    3-player games are FIXP-complete {{[}2007-Etessami{]}}
  \end{itemize}
\end{itemize}

\end{document}
